\documentclass{article}

\usepackage{ulem}
\usepackage{amsmath}
\usepackage{amsfonts}
\usepackage{amsthm}
\usepackage{amssymb}
\usepackage{lipsum}
\usepackage{cancel} 
\usepackage{tikz}
\usepackage{graphicx} 
\usepackage{float} 
\usepackage{pifont}
\usepackage[most]{tcolorbox}

\newtheorem*{theorem}{Teorema}
\newtheorem*{corollary}{Corolário}
\newtheorem*{lemma}{Lema}

\renewcommand{\proofname}{\uline{Demonstração}}

\newcounter{examplecounter}
\newenvironment{exemplo}{\begin{quote}%
    \refstepcounter{examplecounter}%
  \textbf{Exemplo \arabic{examplecounter}}%
  \quad
}{%
\end{quote}%
}

\newtcbtheorem{exercise}{Exercício}%
{ % frame stuff
    enhanced,frame empty,interior empty,
    colframe=blue,
    borderline west={1pt}{0pt}{green!25!blue},
    left=0.2cm,
    % title stuff
    attach boxed title to top left={yshift=-2mm,xshift=-2mm},
    coltitle=black,
    fonttitle=\bfseries,
    colbacktitle=white,
    boxed title style={boxrule=.4pt,sharp corners}}{exercise}

\title{Trabalho 1 - ACH2012 Cálculo II}
\author{
  André Caravalho\\
  \texttt{Número USP: 13672425}
  \and
  Alex Zhu Guangxiao\\
  \texttt{Número USP: 13672130}
  \and
  Cayo Felipe Barbosa Félix Soares\\
  \texttt{Número USP: 13719520}
  \and
  Christiano Bernini Peres\\
  \texttt{Número USP: 5198251}
  \and
 Danilo Araujo de Oliveira Romeira\\
  \texttt{Número USP: 13725823}
  \and
 Eduardo Crispim de Moraes\\
  \texttt{Número USP: 14567051}
  \and
 Egydio Pacheco Neto\\
  \texttt{Número USP: 9849160}
  \and
 Felipe Akira Dias\\
  \texttt{Número USP: 13673051}
  \and
 Frederico Sant'Anna Kimura\\
  \texttt{Número USP: 13831526}
  \and
 Gabriel Chen Hirata\\
  \texttt{Número USP: 13747946}
  \and
 Helena Cota Furtado\\
  \texttt{Número USP: 13835173}
  \and
 João Victor Andrade Lucio\\
  \texttt{Número USP: 11207877}
  \and
 Júlia Du Bois Araújo Silva\\
  \texttt{Número USP: 14584360}
  \and
 Pedro Henrique Resnitzky Barbedo\\
  \texttt{Número USP: 14657691}
  \and
 Rafael Francisco de Freitas Timoteo\\
  \texttt{Número USP: 12924740}
  \and
 Rafael Moura de Almeida\\
  \texttt{Número USP: 11225505}
  \and
 Samuel Antonio Silva\\
  \texttt{Número USP: 13903209}
}
\date{}

\begin{document}

\maketitle
\pagebreak

% ===========================PAG_1===============================

\begin{center}
\section*{Equações Quadráticas}
\end{center}

\[
    x^2 + bx = c \Rightarrow, x = ?
\]

\begin{tikzpicture}

\draw (0,0) rectangle (4,4); % Desenha um quadrado de (0,0) a (4,4)
\node at (2,2) {$X^2$}; % Adiciona a etiqueta "X²" ao centro do quadrado
\node at (2,4.25) {$X$}; % Adiciona a etiqueta "X" acima do quadrado

\draw (4,0) rectangle (6,4); % Desenha um retângulo à direita do quadrado
\node at (5,4.25) {$b$}; % Adiciona a etiqueta "b" acima do retangulo
\node at (5,2) {$bX$}; % Adiciona a etiqueta "bX" ao centro do retangulo

\draw (5,0) -- (5,4); % Desenha uma linha vertical para separar o retângulo à direita ao meio

\draw (0,0) rectangle (4,-1); % Desenha um retângulo abaixo do quadrado
\node at (-0.25,-0.5) {\(\frac{{b}}{2}\)}; % Adiciona a etiqueta "b/2" ao lado esquerdo do retângulo inferior
\end{tikzpicture}

\[
c = x^2 + bx
\]

\[
= (x+b/2)^2 - (b/2)^2
\]

\[
\therefore (x+b/2)^2 = (b/2)^2 + c
\]

\[
x + b/2 = \sqrt{(b/2)^2 + c}
\]

\[
\boxed{x =  -b/2 + \sqrt{(b/2)^2 + c}} = -b/2 + \frac{\sqrt{(b^2 + 4c}}{2}
\]

\[
(\frac{b}{2})^2 + c = \frac{b^2}{4} + c = \frac{b^2 + 4c}{4}
\]

\[
ax^2 + bx = c \Rightarrow x = ax^2 + bx - c \Rightarrow
\]

\[
a^2x^2 - abx = -ac
\]

\[
ax^2 + b(ax) = -ac
\]


% ===========================PAG_2===============================

\[
ax = \frac{-b}{2} + \frac{\sqrt{(b^2 - 4ac}}{2} \Rightarrow \boxed{x = \frac{-b + \sqrt{(b^2 - 4ac}}{2a}}
\]

Katz.pp.23-5. 
5000 a.c. 
1700 a.c. Hammurabi

\[
(ax + b/2)^2 - [(\frac{b}{2})^2 - ac]
\]

\pagebreak

\begin{tikzpicture}

\draw (0,0) rectangle (4,4); % Desenha um quadrado de (0,0) a (4,4)
\node at (2,4.25) {$X$}; % Adiciona a etiqueta "X" acima do quadrado

\draw (4,4) -- (6,4); % linha que representa Y
\node at (5,4.25) {$Y$}; % Adiciona a etiqueta "Y" acima da linha

\draw (0,2) -- (4,2); % Desenha uma linha horizontal para dividir o quadrado ao meio na horizontal
\node at (-0.25,2) {\(\frac{{b}}{2}\)}; % Adiciona a etiqueta "b/2" a esquerda do quadrado

\draw (0,2) rectangle (3,4); % Desenha um retangulo de (0,2) a (3,4)
\node at (0.25,3) {$Y$}; % Adiciona a etiqueta "Y" 
\node at (1.5,3.7) {\(\frac{{b}}{2}\)}; % Adiciona a etiqueta "b/2" 

\draw (0,2) rectangle (2,1); % Desenha um retangulo de (0,2) a (2,1)
\node at (-0.5,1) {\(\frac{{X - Y}}{2}\)}; % Adiciona a etiqueta "(X-Y)/2" 

\draw[dashed] (2,1) rectangle (3,2); % Desenha um quadrado de linha tracejada nas coordenadas (2,1) e (3,2)

\draw (3,-0.15) -- (3,0.15); 
\node at (3,-0.5) {\(\frac{{X - Y}}{2}\)}; % Adiciona a etiqueta "(X-Y)/2" 

\end{tikzpicture}

\[
x + y = b
\]
\[
xy = c
\]

\[
Se x = y \Rightarrow
\begin{cases}
    \ 2x + b \Rightarrow x = b/2\\
    \ x^2 = c \Rightarrow (\frac{b}{2})^2 = c\\
    \ b^2 = 4c.
\end{cases}
\]

Suponha $x > y$

\[
b = x + y = \overline{x} + 2y - \overline{y}
\]

\[
\therefore \frac{b}{2} = y + \frac{x - y}{2}.
\]

\[
b = x + y = 2x - x + y
\]

\[
\therefore \frac{b}{2} = x - \frac{x - y}{2}
\]

\[
x - b/2 = \frac{x - y}{2}
\]

\[
c = xy = (\frac{b}{2})^2 - (\frac{x - y}{2})^2
\]

\[
= (\frac{b}{2})^2 - (x - b/2)^2
\]

\[
\therefore (x - b/2)^2 = (b/2)^2 - c
\]

\[
x - b/2 = \sqrt{(b/2)^2 - c}
\]

\[
x = + b/2 + \sqrt{(b/2)^2 - c} = \frac{b}{2} + \frac{\sqrt{b^2 - 4c}}{2}
\]

\[
x (b - x) = c
\]

\[
bx - x^2 = c
\]

\[
x^2 - bx + c = 0 
\]

\[
x^2 - bx = -c
\]

% ===========================PAG_3===============================
\pagebreak
Sejam $p, q \in \mathbb{R}$ dados. O problema consiste em encontrar $x$ e $y$ tais que:

\[
\left\{
\begin{array}{cc}
   x +  y = p \\
   x \cdot y = q
\end{array}
\right.
\]

Observe que $x + y = p \Rightarrow y = p - x$. Portanto, $q = xy = x(p - x)$.

Logo, tem-se:
\begin{align*}
&\boxed{x \cdot (p - x) = q} \tag{1} \\
&xp - x^2 = q \leftrightarrow \boxed{x^2 - xp + q = 0} \tag{2}
\end{align*}

\underline{Observação:} Se $x$ é uma solução da equação acima, então $p - x$ também é uma solução. Com efeito, observe que a equação (1) é \underline{invariante} pela substituição de $x$ por $p - x$, ou seja:

\[
q = (p - x) (p - (p-x)) = (p-x)(p - p + x) = x (p - x).
\]

Uma prova alternativa usando (2) pode ser obtida analogamente, ou seja:

\[
(p-x)^2 - (p-x) p + q = (p - x)(\cancel{p} - x - \cancel{p}) + q = q - x (p-x) = 0.
\]
seu $(1) \leftrightarrow (2)$.


\textbf{Raízes}: x = $\frac{p \pm \sqrt{p^2 - 4q}}{2}$

% ===========================PAG_4===============================

Observe que as raízes são reais se $p^2 - 4q \geq 0 \leftrightarrow p^2 \geq 4q \leftrightarrow q \leq \frac{p^2}{4}$. Portanto, o valor máximo de $q$ é $\frac{p^2}{4}$. Graficamente, basta observar que: 
$$\frac{p^2}{4} = \frac{p}{2} \cdot \frac{p}{2}$$.

\begin{figure}[H]
    \centering
    \includegraphics[width=0.6\linewidth]{Figuras_Graficos/Grafico pag 4.png}
    \label{Grafico pag4}
\end{figure}

\pagebreak

\begin{exemplo}
Dentre todos os retângulos de perímetros $P$, existe algum com área máxima?

\begin{tikzpicture}

\draw (0,0) rectangle (4,2); % Desenha um quadrado de (0,0) a (4,4)
\node at (-0.25,1) {$Y$}; % Adiciona a etiqueta "Y" ao lado quadrado
\node at (2,-0.25) {$X$}; % Adiciona a etiqueta "X" abaixo do quadrado

\end{tikzpicture}

\begin{align*}
&P = 2x + 2y = 2(x+y) \\
&A = x \cdot y \\
&\therefore x + y = \frac{P}{2} = p \\
&xy = A = q \therefore q \\
&\therefore \text{ é máximo } \rightarrow x = \frac{p \pm 0}{2} = \frac{p}{2} = \frac{P}{4} \\
&\therefore y = p - x = \frac{P}{2} - \frac{p}{4} = \frac{P}{4} = x. \quad \therefore \text{é um \underline{quadrado}}.
\end{align*}

Observe que: $q_{\text{max}} = x \cdot y = \frac{P^2}{4^2} = \frac{\frac{P^2}{4}}{4} = \frac{\left(\frac{P}{2}\right)^2}{4}
= \frac{p^2}{4}$.
\end{exemplo}

% ===========================PAG_5===============================
\pagebreak
\section*{Forma Canônica}

\begin{theorem}
   Toda função quadrática $f(x) = ax^2+bx+c\;(a\neq0)$ pode ser expressado na forma: $f(x) = a(x-m)^2 + k$, onde $m = -b/2a$ e $k = f(m)$. 
\end{theorem}


\begin{proof}\mbox{}\\*
Basta observar que:
\begin{align*}
f(x) &= a(x^2+ \frac{b}{a}x + \frac{c}{a}) \\
&= a\left(x^2 + 2 \frac{b}{2a}x + \left(\frac{b}{2a} \right)^2 - \left(\frac{b}{2a} \right)^2 + \frac{c}{a}\right) \\
&= a\left(x + \frac{b}{2a}\right)^2 + c - \frac{\cancel{a}b^2}{4\cancel{a^2}} \\
&= a\left(x + \frac{b}{2a}\right)^2 + \frac{4ac - b^2}{4a} \\
&\therefore \;m = -b/2a \text{  e  } k = \frac{4ac - b^2}{4a}
\end{align*}
\end{proof}
$\text{obviamente, } f(m) = a (m-m)^2 + k = 0 + k = k.$


\begin{corollary}
Se $f(x_1) = f(x_2)$ com $x_1 \neq x_2$, então $m = \frac{x_1 + x_2}{2}$.
\end{corollary}

\begin{proof}
\begin{align*}
   &\cancel{a}(x_1-m)^2 + k = f(x_1) = f(x_2) = \cancel{a}(x_2-m)^2 + k \rightarrow \\
   &\rightarrow x1-m = \pm (x_2-m). \text{ Se } x_1-\cancel{m} = x_2-\cancel{m} => x_1 = x_2. \\
   &\text{ Se } x_1-m = -(x_2-m) = -x_2+m => x_1+x_2 = 2m => m=\frac{x_1+x_2}{2}
\end{align*}
\end{proof}

% ===========================PAG_6===============================
\pagebreak
\begin{corollary}
$m = min\;\{f(x): x\in R\}. $ se $a>0$, ou máx se $a>0$.
\end{corollary}


\begin{proof}
\[f(x) = a(\underbrace{x-m}_{\geq 0})^2 + k.\text{ Logo o valor minímo ocorre se } x-m = 0 \Longleftrightarrow x=m. \]
\end{proof}


\begin{corollary}
    $f(2m-x) = f(x).$
\end{corollary}

\begin{proof}
\begin{align*}
    f(2m-x) &= a(\cancel{2}m-x-\cancel{m})^2 + k = a(m-x)^2+k = f(x) \\
    \therefore \; f(m+x) &= f(2m - (m+x)) = f(\cancel{2}m-\cancel{m}-x) \\
    &=f(m-x).
\end{align*}
\end{proof}


Em particular, f é simétrico com relação à linha vertical x=m.

\begin{figure}[H]
    \centering
    \includegraphics[width=0.6\linewidth]{Figuras_Graficos/Grafico pag 6.png}
    \label{Grafico pag6}
\end{figure}

\begin{align*}
    f(m+x) &= a(\cancel{m}+x-\cancel{m})^2 + k = ax^2+k = a(-x)^2+k \\
    &=a(m-x-m)^2 + k = f(m-x).
\end{align*}

% ===========================PAG_7===============================
\pagebreak
\subsection*{\underline{Gráfico: A Parábola}}

\begin{flushleft}
Considere f(x) = $ax^2$, com a $\neq$ 0. \\
Foco: F = (0,k) \\
Diretriz D: y = -k.\\   
\end{flushleft}

\begin{figure}[H]
    \centering
    \includegraphics[width=0.6\linewidth]{Figuras_Graficos/Grafico pag 7.png}
    \label{Grafico pag7}
\end{figure}

\begin{align*}
    x^2 + (y-k)^2 &= d^2 = (y+k)^2\\
    x^2+y^2 - 2yk + k^2 &= y^2 + 2yh + k^2  \\
    x^2 - 2yk &= 2yk \Leftrightarrow 4yk = x^2.\\
\end{align*}
\[
\therefore \space y=ax^2 \implies 4ax^2k = x^2 \implies k = \frac{1}{4a}
\]
Reciprocamente, $4yk = x^2 \implies y = \frac{1}{4k}x^2\implies f(x) = ax^2$, 
com $a=\frac{1}{4k}\Leftrightarrow k = \frac{1}{4a}$. \\
Portanto, o gráfico da função $f(x) = ax^2$, é uma \underline(parábola) com foco $F = (0,k)$ e 
diretriz $y = -k$, com $k=\frac{1}{4a}$.\\
Reciprocamente, toda parábola é gráfico de uma função $f(x) = ax^2$ com $a = \frac{1}{4k}$

% ===========================PAG_8===============================
\pagebreak

\begin{corollary}
    O gráfico de toda função quadrática é uma parábola.
\end{corollary} 

\begin{proof}\mbox{}\\*
Basta usar a forma canônica: $f(x) = a(x-m)^2 + k$ pois $x-m$ é um deslocamento horizontal enquanto que $+k$ é um deslocamento vertical.
\[
g(x) = f(x+m) - k = a(x+m-m)^2 + k - k = ax^2
\]
É uma parábola $\implies g(x)+k=f(x+m)$ é a mesma parábola, deslocada verticalmente. $f(x) = g(x-m)+k$ é a mesma parábola, deslocada horizontalmente.
\end{proof}

Define-se a \underline(parábola) como o lugar geométrico dos pontos(x,y) 
no plano que são equidistantes de uma reta D e um ponto F tal que F 
$\notin$ D.\\
Essa reta denomina-se \underline{diretriz} de parábolas, e que o ponto F é 
o seu \underline{foco}.



\begin{figure}[H]
    \centering
    \includegraphics[width=0.6\linewidth]{Figuras_Graficos/Grafico pag 8.png}
\end{figure}

% ===========================PAG_9===============================
\pagebreak

Considere uma amostra de tamanho $N$ composta por apenas dois valores $a$ e $b$, $a \neq b$, com frequências respectivas $n$ e $N-n$, com $0 \leq n \leq N$.
 
Em tal caso a média é dada por:
 
\[ M = M(n) = \frac{n . a + (N-n) b}{N} = b - \frac{n(b-a)}{N} \textnormal{, função linear}\]
Em particular, tem-se:
\[\ M(1) = b - \frac{(b-a)}{N} \approx b  \textnormal{, se } N>> b-a\]
Como também: 
\[M(n) = b-(b-a) = a\]
Por outro lado, o desvio padrão $s$ é dado por: 
\begin{align*}
(N-1)s^2 &= n(a-n)^2 + (N-n) (b-n)^2 \\
 &= n (a-b + \frac{n}{N}(b-a))^2 + (N-n)(b-b+\frac{n}{N}(b-a))^2 \\
 &= n (b-a)^2 (\frac{n}{N}-1)^2 + (N-n) (b-a)^2 (\frac{n}{N})^2 \\
 &=(b-a)^2 [n((\frac{n}{N})^2 - 2\frac{n}{N}+1) +\frac{Nn^2}{N^2} -  n(\frac{n}{N})^2] \\
 &=(b-a)^2 [n(\frac{n}{N})^2  - 2\frac{n}{N} +n +\frac{n}{N} -n (\frac{n}{N})^2]\\
 &= - \frac{(b-a)^2}{N} (n^2 - nN) \\ 
\therefore \frac{N(N-1)}{(b-a)^2}s^2 &= {-(n^2 - 2\frac{nN}{2} + (\frac{N}{2})^2-(\frac{N}{2})^2)} \\
 &= - (n - \frac{N}{2})^2 + (\frac{N}{2})^2
\end{align*}

\[
 \therefore \boxed{\frac{N(N-1)}{(b-a)^2}s^2 + (n - \frac{N}{2})^2 = (\frac{N}{2})^2}
\]

% ===========================PAG_10===============================

\pagebreak

\textbf{Desta maneira, resulta uma \underline{elipse} no plano (n,s):}


\[
\frac{s^2}{\frac{N^2}{4}\frac{(b-a^2)}{N(N-1)}} + (\frac{(n-\frac{n}{2}^2)}{(\frac{N}{2})^2}) = 1.
\]

\[
\frac{S^2}{\frac{N(b-a)^2}{4(N-1)}} + \frac{(n-\frac{N}{2})^2}{(\frac{N}{2})^2} = 1.
\]

\begin{figure}[H]
    \centering
    \includegraphics[width=0.6\linewidth]{Figuras_Graficos/Grafico pag 10.png}
\end{figure}

Ponto extremo (máximo): $n=\frac{n}{2}$

Valor máximo: $\frac{N(N-1)s^2}{(b-a)^2}$ = $(\frac{N}{2})^2$ = $(\frac{N}{2}^2)$ = $\frac{N^2}{4}$ $\Rightarrow{}$ $s = \frac{b-a}{2}\sqrt{\frac{N}{N-1}}$.


\textbf{Definindo o desvio padrão com N em lugar de N - 1, tem-se:}

\begin{align*}
    &\frac{(2s)^2}{(b-a)^2} + \frac{(2n - n)^2}{N^2} = 1. \quad \text{Definindo $x = 2n$ e $y = 2s$,} \\
    &\text{resulta a elipse: $\frac{(x-n)^2}{N^2} + \frac{y^2}{(b-a)^2} = 1$.}  
\end{align*}


% ===========================PAG_11===============================
\pagebreak

\begin{theorem}
    Seja $F$ diferenciável em a com $f'(a)>0$. Então existe $\delta > 0$ tal que: x $\in$ (a, a+$\delta$) $\Rightarrow{} f(x)>f(a)$ e \quad x $\in$ (a-$\delta$, a) $\Rightarrow{} f(x)<f(a)$.
\end{theorem} 

\begin{proof}\mbox{}\\*
    Seja $\epsilon:= \frac{f'(a)}{2}>0$. \\
    Por hipótese, existe $\delta>0$ tal que:
\begin{align*} 
    0 < |x - a| < \delta \Rightarrow |\frac{f(x) - f(x)}{x-a} - f'(a)| < \epsilon \\
    \therefore -\epsilon < \frac{f(x) - f(a)}{a - a} - f'(a) < \epsilon. \\
    f'(a) - \epsilon < \frac{f(x) - f(a)}{x - a} < f'(a) + \epsilon \\
    \frac{f(x) - f(a)}{x - a} > f'(a) - \epsilon = f'(a) -\frac{f'(a)}{2} = \frac{f'(a)}{2} > 0 \\
    \therefore a<x<a+\delta \Rightarrow f(x) > f(a) \\
    \text{e} \quad a - \delta < x < a \Rightarrow f(x) < f(a).
\end{align*}
\end{proof}

% ===========================PAG_12===============================
\pagebreak

\begin{exemplo}
\[
\text{Considere} \quad F(x) =
\begin{cases}
    \ x/2 + x^2sen(1/x) ; \quad x \neq 0\\
    \ 0 \quad  {se} \quad x = 0\\
\end{cases}
\]
\end{exemplo}

\begin{align*}
    Se \ x \neq 0 \ tem-se: \textit{f'}(x) = \frac{1}{2} + 2x.sen\frac{1}{x} - \frac{\cancel{x^2}cos\frac{1}{x}}{\cancel{x^2}}\\
    se \ x=0 \ \ \ tem-se: \textit{f'}(0) = \displaystyle\lim_{x \to 0} \frac{f(x) - f(0)}{x}\\
    se \ x=0 \ \ \ tem-se: \textit{f'}(0) = \displaystyle\lim_{x \to 0} \frac{f(x) - f(0)}{x}\\
    \lim_{x\to 0} \frac{1}{2} + x.sen \frac{1}{x} = \frac{1}{2} + \underbrace{\lim_{x\to 0} x.sen \frac{1}{x}}_{= \ 0} = \frac{1}{2}
\end{align*}
\[0 \ \leq \ | \ x.sen \ \frac{1}{x} \ | \ \leq \ |x|\]

$\therefore$ \textit{f'} é diferenciável em $\mathbb{R}$ com $f'(0) = \frac{1}{2} > 0$.
Observe que \textit{f'} não é contínua na origem pois $cos \ \frac{1}{x}$ diverge quando $x \rightarrow 0.$ \ Alternativamente, observe que para $x_n = \frac{1}{2n \pi } \pi $

tem-se: 
\[\left.\
\begin{array}{c}
sen \frac{1}{x_n} = sen 2n \pi = 0\\
cos \frac{1}{x_n} = cos 2n \pi = 1 
\end{array}
\right\}\forall n \in \mathbb{R}\]

\begin{align*}
    \therefore \  \textit{f'} (x_n) = \frac{1}{2} - 1 = - \frac{1}{2} \ \  \forall n \in \mathbb{N}\\
    \therefore x_n \rightarrow 0, \text{mas}, f' (x_n) = -\frac{1}{2}  \rightarrow \textit{f'}(0) = \frac{1}{2}\\
\end{align*}

% ===========================PAG_13===============================
\pagebreak

\[
g(x) = \frac{x}{2} + x^2 = x(x +\frac{1}{2})
\]
\[
h(x) = \frac{x}{2} - x^2 = -x(x -\frac{1}{2})
\]


\begin{figure}[H]
    \centering
    \includegraphics[width=0.6\linewidth]{Figuras_Graficos/Grafico pag 13.png}
\end{figure}

\begin{align*}
    g'(x) = \frac{1}{2} + 2x \Rightarrow g'(0) = \frac{1}{2}\\
    h'(x) = \frac{1}{2} - 2x \Rightarrow h'(0) = \frac{1}{2}
\end{align*}

$\therefore$ Em qualquer intervalo (-a,a) com $a>0$ \textit{f} \underline{não} é monótona   

\begin{align*}
    \underbrace{2n \pi}_{\frac{1}{d}} \ < \ \underbrace{2n\pi + \frac{\pi}{2}}_{\frac{1}{c}} \ < \ \underbrace{2(n+1) \pi}_{\frac{1}{b}} \ < \ \underbrace{2(n+1) \pi + \frac{\pi}{2}}_{\frac{1}{a}}\\
    sen \frac{1}{d} = \ \ sen \frac{1}{b} = 0; \ \  sen \frac{1}{c} = \ \ sen \frac{1}{a} = 1
\end{align*}

\[f(a) = \frac{a}{2} + a^2; \ \ \ \ \ \ f(c) = \frac{c}{2} + c^2\]
\[f(b) = \frac{b}{2}; \ \ \ \ \ \ \ \ f(d) = \frac{d}{2}   \]

% ===========================PAG_14===============================
\pagebreak

\subsection*{Teorema de Darboux:}

Seja $f: [a, b] \rightarrow \mathbb{R}$ uma função diferenciável com $f'(a) < d < f'(b)$.\\
Então, existe $c \in (a, b)$ tal que $f'(c) = d$.

\begin{proof}\mbox{}\\*
Seja $g(x) = f(x) - dx$.

então, $ g'(x) = f'(x) - d$, com $g'(a)<0<g'(b)$.

Pelo Teorema anterior, existe $\delta > 0$ tal que:

\[
x \in (b-\delta,\delta) \Rightarrow g(x) < g(b)
\]
Como também:
\[
x \in (a,a+\delta) \Rightarrow  g(x) < g(a)
\]

Agora, g contínua $\Rightarrow$ atinge um valor mínimo, ou seja, existe $c \in [a,b]$ tal que $g(c) \leq g(x), \forall x \in [a,b]$

\begin{center}
$(1) + (2) \Rightarrow c \notin [a,b] \Rightarrow c \in (a,b)$.
\end{center}

Como $g$ é diferenciável, deve ser 
\[
g'(c) = 0 \therefore 0 = g'(c) = f'(c) - d \Rightarrow f'(c) = d
\]
\end{proof}


% ===========================PAG_15===============================
\pagebreak

\[
    S_n = 1 + a + a^2 + ... + a^n = \sum_{i=0}^{n} a^i
\]    

Multiplicando a equação por $a$ temos:

\[
    aS_n = a + a^2 + ... + a^n + a^{n+1}
\]

Ao realizar a subtração das duas equacões temos:

\[
    aS_n = 1 + a^{n+1}
\]

\[
    \therefore (1-a)S_n = 1-a^{n+1} \Rightarrow
\]

\[
    \Rightarrow  S_n = \frac{1-a^{n+1}}{1-a} \Rightarrow
\]

\[
    \Rightarrow \frac{1}{a-1} = \underbrace{1 + a + ... + a^n}_{S_n = \sum_{i=0}^{n} a^n} + \frac{a^{n+1}}{1-a}
\]

Trocando a por $-x^2$, tem-se:

\[
\frac{1}{1+x^2} = \sum_{i=0}^{n} (-x^{2})^i + \frac{(-x^2)^{n+1}}{1+x^2}
\]    

Ou

\[
  \frac{1}{1+x^2}  = \sum_{i=0}^{n} [(-1)^i+x^{2i}] + \frac{(-1)^{n+1}x^{2n+2}}{1+x^2}
\]

\begin{theorem}
    $arctg x = \underbrace{\sum_{i=0}^n[(-1)^i \frac{x^{2i+1}}{2i+1}]}_{=P_{2n+1(x)}}  + R_{2n+1}(x)$
\end{theorem}

\begin{proof}
    Basta provar que: $\lim_{x \rightarrow 0} \frac{arctgx - P_{2n+1}(x)}{x^{2n+1}} = 0$. \\
    Usando a regra de L'Hôpital, tem-se:

    \begin{align*}
    \frac{arctan'x - P_{2n+1}'(x)}{(2n+1)x^{2n}} &= \frac{1}{(2n+1)x^{2n}}\left[\frac{1}{1+x^2} - \sum_{i=0}^n{(-1)^i x^{2i}}\right] \\
    &= \frac{(-1)^{n+1} x^{2n+2}}{(2n+1) x^{2n} (1+x^2)} \\
    &= \frac{(-1)^{n+1} x^2}{(2n+1)(1+x^2)}\frac{(-1)^{n+1} x^2}{(2n+1)(1+x^2)} \rightarrow 0. 
    \end{align*}
\end{proof}

% ===========================PAG_16===============================
Por outro lado, trocando $a$ por $-x$, tem-se:

$$
\frac{1}{1+x} = \sum^n_{i=0}(-x)^i + \frac{-x^{n+1}}{1+x} = \sum^n_{i=0}(-1)^ix^i+\frac{-1^{n+1}x^{n+1}}{x+1}
$$

\begin{theorem}
\[ ln(1+x) = \sum^n_{i=0}(-1)^i\frac{x^{i+1}}{i+1}+R_{n+1}(x) \]
\end{theorem}

\begin{proof}\mbox{}\\*
Usando a regra de L'Hôpital, tem-se: 
\[
\frac{\ln'(1+x) - P'_{n+1}(x)}{(n+1)x^n} = \frac{1}{(n+1)x^n}\left[\frac{1}{1+x} - \sum_{i=0}^{n}(-1)^ix^i\right] =
\]

\[
\frac{(-1)^{n+1}x^{n+1}}{(n+1)x^n(1+x)} = \frac{(-1)^{n+1}x}{(n+1)(1+x)} \rightarrow 0 \qedhere
\]
\end{proof}



\begin{exercise}{Aproximação $\pi$}{ex:mylabel}
Usando a expressão integral para o resto, aproximando a integral por somas de Riemann, demonstre correções para $\pi$ do tipo madhava (escola de Kerala): Resto de $\arctan$ igual a $(-1)^nF(n)$, onde $F=F_i$ com:
$$
F_1(n) = \frac{1}{4n},
F_2(n) = \frac{n}{4n^2+1},
F_3(n) = \frac{n^2 + 1}{4n^3+5n},
$$
\end{exercise}

% ===========================PAG_17===============================
\pagebreak

\begin{lemma}
Se $\lim\limits_{x \to \infty} f(x+1) - f(x) = k$, então $\lim\limits_{x \rightarrow \infty}\frac{f(x)}{x} = k$.
\end{lemma}

\begin{proof}\mbox{}\\
Seja $\epsilon > 0$. Por hipótese, $\exists N=N(\epsilon)$ tal que:

\[ 
x \geq N \implies |f(x+1) - f(x) -k| < \epsilon
\]

\[
k - \epsilon < f(x+1) - f(x) < k + \epsilon
\]

Em particular, tomando $x=N, N+1, \ldots, N+n$, com $n \in \mathbb{N}$
. Tem-se:

\[
  n(k - \epsilon) < \sum_{i=0}^{n-1}f(N+i+1) + f(N+i) < n(k+\epsilon)
\]

\[
  -\epsilon < \frac{f(N+n) - f(N)}{n} - k < \epsilon
\]

e tem-se $|\alpha|<\epsilon$ $\forall n \in \mathbb{N}$. Agora:

\[
\frac{f(N+n)-f(N)}{n}-k = \alpha \implies f(N+n) = f(N) + n(\alpha+k)
\]

\[
f(x) = f(N) + (x-N)(\alpha+k)
\]

\[
\therefore \frac{f(x)}{x} = \frac{f(N)}{x}+(1-\frac{N}{x})(\alpha+k)
\]

\[
\frac{f(x)}{x}-k = \frac{f(N)}{x} + (1 - \frac{N}{x})\alpha - \frac{N}{x}k
\]

\[
\therefore |\frac{f(x)}{x} - k| \le c |\alpha| < \epsilon \qed
\]

\end{proof}

% ===========================PAG_18===============================
\pagebreak

\begin{exemplo}
    \[
    log(x+1) - log{x} = log(\frac{x+1}{x}) = log(1+ \underbrace{\frac{1}{x}}_{\rightarrow{0}}) 
    \]
    \[
    \text{converge para} \log1 = 0 \quad \text{quando} \quad x \rightarrow{+\infty}.
    \]
    \[
    \therefore \frac{\log{x}}{x} \rightarrow{0}\; \text{se}\; x \rightarrow{+ \infty}
    \]
    
\end{exemplo}



\underline{Observação}: O mesmo resultado do lema vigora se em lugar de $\rightarrow{k}$ tem-se $\rightarrow{\pm\infty}$. Com efeito, dado

\[M>0\;\; \exists\; N = N(\varepsilon) \in \mathbb{N}\; \text{tal que}\; x \geq N \Rightarrow f(x+1)-f(x)>M\]

\[\therefore \underbrace{\frac{f(N+n)-f(N)}{n}}_{=: \alpha} >n\;\; \forall n \in \mathbb{N}.\]

\[\frac{f(N+n)-f(N)}{n} = \alpha \Rightarrow f(N+n) = f(N) + \overbrace{n\alpha}^{= x-N} \Rightarrow\]

\[\frac{f(x)}{x} = \underbrace{\frac{f(N)}{x}}_{\rightarrow{0}} + \underbrace{(1-\frac{N}{x})}_{\rightarrow{1}} \underbrace{\alpha.}_{>n\; \forall n} \qed\]

 
\begin{exemplo}
    $e^{x+1}-e^{x} = (e-1)e^{x} \rightarrow + \infty\; \text{se}\; x \rightarrow +\infty$.
\end{exemplo}


\[\therefore \lim_{x \rightarrow +\infty} \frac{e^{x}}{x} = +\infty.\]


Observe que tomando $x = e^{y}$ no exemplo anterior tem-se:

\begin{equation*}
    \frac{\log{e^{y}}}{e^{b}} = \frac{y}{e^{y}} \rightarrow 0\; se\; y \rightarrow +\infty
\end{equation*}

% ===========================PAG_19===============================
\pagebreak

\subsection*{Seja $f:\mathbb{R} \rightarrow \mathbb{R}$ com as seguintes propriedades:}

\begin{enumerate}
\item Monotonicidade: $x<y \Rightarrow f(x) < f(y)$
\item Morfismo aditivo: $f(x+y) = f(x) + f(y)\;\; \forall x, y \in \mathbb{R}$
\end{enumerate}

\subsection*{1. Prove que $f(0) = 0$.}

\[f(0) = f(0+0) = f(0)+f(0) = 2f(0)\]

\[\therefore f(0) \neq 0 \Rightarrow 1 = 2,\; \text{absurdo!}\; \therefore\; \text{deve ser}\; f(0) = 0.\]

\subsection*{A. Prove que $f$ é contínua em $\mathbb{R} \Leftrightarrow f$ é contínua na origem.}

\[(\cancel{a}) \stackrel{?}{=} \lim_{h \rightarrow 0} f(a+h) = \lim_{h \rightarrow 0} f(a)+f(h)\]

\[= f(\cancel{a}) + \lim_{h \rightarrow 0} f(h) \Leftrightarrow \lim_{h \rightarrow 0} f(h) = 0.\]

\subsection*{2. Prove que $f(m) = mf(1)\; \forall m \in \mathbb{N}$}

\[f(m) = f \underbrace{(1+...+1)}_{m\; vezes} = f(1)+...+f(1) = mf(1).\]

\begin{corollary}
$f(1) \neq 0$
\end{corollary}


\begin{proof}\renewcommand{\qedsymbol}{}
\[0<x<m \Rightarrow 0 = f(0) < f(x) < f(m) = mf(1) = 0.\]
Seja $a := f(1) \neq 0.$
\[x<1+x \Rightarrow f(x) < f(1)+f(x) = a+f(x) \Rightarrow \fbox{$0<a$}\]
\end{proof}

% ===========================PAG_20===============================
\pagebreak
\subsection*{3. Prove que $f( \frac{1}{n} )  = \frac{a}{n} \ \ \forall \ n \in \mathbb{N}:$ }
\begin{align*}
a &= f(1) = f(\frac{n}{n}) = f(\underbrace{\frac{1}{n} + ... + \frac{1}{n}}_\text{n vezes}) = f(\frac{1}{n}) + ... + f(\frac{1}{n}) \\
 &= n \ f(\frac{1}{n}) \ \therefore \ f(\frac{1}{n}) = \frac{a}{n}
\end{align*}

\subsection*{B. Prove que $f$ é contínua na origem:}
\[
\lim_{u\rightarrow 0} f(u) = f(0) = 0 \Leftrightarrow \forall \ \varepsilon > 0 \ \exists \ \delta > 0 : 0 < \left | x \right | < \delta \Rightarrow \left | f(x) \right | < \varepsilon
\]
\[
0 < x < \frac{1}{n} \Rightarrow 0 = f(0) < f(x) < f(\frac{1}{n}) = \frac{a}{n} \overset{?}{<} \varepsilon
\]
\[
\text{Dado} \ \varepsilon > 0 \ \text{seja} \ n \in \mathbb{N}: n > \frac{a}{\varepsilon }
\]
\[
\therefore \ \text{Basta tomar} \ \delta = \frac{1}{n}
\]
\[
0 = f(0) = f(x-x) = f(x) + f(-x) \Rightarrow f(-x) = -f(x)
\]
\[
\therefore -\frac{1}{n} < x < 0 \Rightarrow 0 < -x < \frac{1}{n} \Rightarrow \left | f(x) \right | = \left | -f(x) \right | = \left | f(-x) \right | < \varepsilon
\]

\vspace{1em}
\subsection*{4. Prove que $f(r) = ar \ \ \forall \ r \in \mathbb{Q}:$} Basta combinar 1, 2, 3.
\[f(\frac{m}{n}) = f(\underbrace{\frac{1}{n} + ... + \frac{1}{n}}_\text{m vezes}) = m f(\frac{1}{n}) = \frac{ma}{n}.\]

\vspace{1em}
\subsection*{5. Prove que $f(x) = ax \ \ \forall \ x \in \mathbb{R}:$}
\[
f(x) < ax \Rightarrow \frac{f(x)}{a} < x \therefore \exists \ r \in \mathbb{Q}: \frac{f(x)}{a} < \underbrace{r < x}_{\Rightarrow \ \fbox{$f(r) <  f(x)$}\ \circledast}
\]
\[
f(x) < ar \overset{\text{\textcircled{4}}}{=} f(r) \ \text{contradiz} \circledast. \
\]

\[
\text{Analogamente prova-se que} \ ax < f(x) \ \text{conduz a contradição.}
\]

\[
\therefore \text{deve ser} \ f(x) = ax
\]

% ===========================PAG_21===============================
\pagebreak

\scalebox{2}{\ding{172}} \textbf{Vide: RUDIN: Princ. Math. Analysis Capítulo 5 Exercício 22}

Seja $f$ definida em $\mathbb{R}$. Se  $f$ é diferenciável com $f'(t) \neq  1  \forall  t \in x$, então $f$ possui \underline{no máximo} apenas um ponto fixo, ou seja, um ponto x tal que $f(x) = x$.

\begin{proof}
 Por redução ao absurdo suponha que existem $u$ e $v$ pontos fixos com $u \neq v$. Pelo Teorema do Valor Médio existe $g \in (u,v)$ tal que 
 \[f'(g) = \frac{ f(v) - f(u)} { v - u} = \frac{v - u}{v - u} = 1\] que  caracteriza um absurdo
\end{proof}

Uma função se diz de Lipchitz se existe $\lambda \geq 0$ tal que,

\[
|f(x) = f(y)| \leq \lambda |x-y| ,  \forall x,y \in \mathbb{R}
\].

O menor de tais $\lambda$ se diz a constante de Lipchitz de $f$. Se $\lambda < 1$, então $f$ se diz uma \underline{contração}.

\begin{exemplo}
Se $f$ é diferenciável e existe $A < 1$ tal que $|f'(t)| \leq A \forall t \in \mathbb{R}$, então $f$ é uma contração. Com efeito, pelo Teorema do Valor Médio, existe $\delta \in (x,y)$ tal que: 
\end{exemplo}

\[
\frac{|f(y) = f(x)|}{|y-x|} = |\frac{f(b)-f(x)}{y-x}| = |f'(x)| \leq A \xrightarrow{}
\]

\[
\xrightarrow{}  |f(y) - f(x)|  \leq A - |y - x|
\].

% ===========================PAG_22===============================
\pagebreak
\scalebox{2}{\ding{173}}

\paragraph{}
Em um espaço métrico \(f\) é uma contração se:

\[
d( f(x), f(y) ) \leq A \cdot d(x, y).
\]

Observe que:
\[
d(x, y) \leq d( x, f(x) ) + \underbrace{d( f(x), f(y) )}_{\leq A \cdot d(x, y)} + d ( f(y), y ).
\]

\[
\therefore\quad d(x, y) \leq \frac{d(x, f(x)) + d(f(y), y)}{1 - A}. 
\]

Em particular, se $x$ e $y$ são pontos fixos, ou seja $f(x) = x$ e $f(y) = y$, então:

\[
0 \leq d(x, y) \leq \frac{\overbrace{d(x, x)}^{=0} + \overbrace{d(y, y)}^{=0}}{1-A}. \quad \Rightarrow
\]

\[
\Rightarrow \quad d(x, y) = 0 \Rightarrow x = y  \quad \therefore \quad \text{Um ponto fixo deve ser único.}
\]

Seja \(x_n\) definida, $x_0$ é arbitrário e $\forall n \in \mathbb{N}$.

\[
\ x_n = f(x_n -1).
\]

\begin{theorem}
    Existe $x := \lim_{n \to \infty} x_n$ e $x$ é um ponto fixo de $f$.
\end{theorem}
    
\[
x_n = f(x_{n-1}) = f^2(x_{n-2}) = \ldots = f^n(x_0).
\]

\begin{align*}
    \therefore\quad d(x^n, x^m) &\leq d(f^n(x_0), f(f^n(x_0)) + d(f(f^m(x_0)), f^m(x_0)).{x\frac{1}{1-A}} \\
    &\leq \mathrm{Ad}(x^{n-1}, x^n) + \mathrm{Ad}(x^m, x^{m-1}). \\
    &\leq A^n \mathrm{d}(x_0, f(x_0)) + A^m \mathrm{d}(f(x_0), x_0). \\
    &\leq \frac{A^n + A^m}{1 - A} \quad \mathrm{d}(x_0, f(x_0)). \\
    &\quad \rightarrow \quad 0, \quad \text{pois } A^n, A^m \rightarrow 0 \text{, dado que } A < 1.
\end{align*}

% ===========================PAG_23===============================
\pagebreak

\begin{lemma}
$\forall n \in \mathbb{N}$ tem se:
\[
d(x_{n-1},x_n)\leq A^n d(f(x_0),x_0)
\]
\end{lemma} 

\begin{proof}\mbox{}\\
Por indução em $n$. Se $n=0$ , então:
\[
d(x_1,x_0)=d(f(x_0),x_0)=A^0 d(f(x_0),x_0).
\]
Agora, 

\begin{align*}
d(x_{n+2},x_{n+1}) &= d(f(x_{n+1}),f(x_n)) \\
&\leq A d(x_{n+1},x_n) \\
&\leq A \cdot A^nd(f(x_0),x_0)  \\
&= A^{n+1}d(f(x_0),x_0). \\
\end{align*}
\end{proof}

\begin{lemma}
Seja $n \in \mathbb{N}$. Então $\forall k \in \mathbb{N}$ tem se: 
\[
d(x_{n+k},x_n) \leq \frac{A^n}{1-A}d(f(x_0),x_0).
\]
\end{lemma}

\begin{proof}
\begin{align*}
d(x_{n+k},x_n)&\leq d(x_{n+k},x_{n+k-1})d(x_{n+k-1},x_{n+k-2})+...+\\
&+d(x_{n+k-(k-1)},x_{n+k-k})\\
\xrightarrow{lema\ \text{anterior}}&\leq \underbrace{(A^{n+k-1}+A^{n+k-2}+\ldots+A^n)}_{\text{$A^n(A^{k-1}+A^{k-2}+\ldots+1) = A^n\frac{1-A^k}{1-A}$}}d(f(x_0),x_0)\\
&\leq\frac{A^n}{1-A}d(f(x_0),x_0).
\end{align*}
\end{proof} 

% ===========================PAG_24===============================
\pagebreak

\begin{corollary}
    ${x_n}$ é de Cauchy
    Se  $m > n$ , então $m = n + k$.
\end{corollary}


\begin{equation*}
\therefore d({x_m},{x_n}) \leq \frac{A^n}{1-A} \quad d({f(x_0)}, {x_0}) \rightarrow 0 \quad \text{se} \quad n \rightarrow \infty
\end{equation*}

Logo, existe o limite $ x = \lim_{n \to \infty}{x_n}$ ,  e tem-se que:

\begin{gather*}
f(x) = f(\lim_{n \to \infty}{x_n}) \underset{\text{f contínua}}{=} \lim_{n \to \infty}{f({x_n})} = \lim_{n \to \infty}{x_{n+1}} = x
\end{gather*}

Observação : tomando para $m \rightarrow \infty $ na sua identidade:
\begin{align*}
&d({x_m},{x_n}) \leq \frac{A^n}{1-A} d({f(x_0)}, {x_0}) \quad \text{tem-se:} \\
&d({x},{x_n}) \leq \frac{A^n}{1-A} d({f(x_0)}, {x_0}) \quad \text{taxa de convergência}
\end{align*}

\underline{Unicidade}: Se $x = f(x) \quad \textbf{e} \quad y = f(y)  \quad  \text{então :}$
\[
d(x,y) = d(f(x),f(y)) \leq  A d(x,y)
\]
\[
\text{Se}\quad d(x,y) \neq 0 \rightarrow A \geq 1 \quad \text{absurdo} \quad \therefore d(x,y) = 0 \rightarrow x = y
\]

\end{document}